\chapter{Понятие L-графов (как лучше назвать?)} \label{chapt2}

\section{Определение L-графа} % (fold)
%\label{sect2_1}
\mydefinition{Пусть $\Sigma_($ и $\Sigma_)$ --- непересекающиеся алфавиты, и существует биективное отображение $\phi:\Sigma_( \rightarrow \Sigma_)$. Тогда назовём непустое множество $P \subseteq \Sigma_( \times \Sigma_)$ \textit{D-множеством}.}

\mydefinition{Пуст $P$ - \textit{D}-множество. Тогда назовём язык $L_P$, порождаемый грамматикой $S \rightarrow \Lambda | aSbS, (a,b)\in P$, \textit{D-языком} (над \textit{D}-множеством $P$).}

\mydefinition{
	Пусть $\Delta = \Sigma_( \cup \Sigma_)$. Определим отображение $\mu: \Delta^* \rightarrow \Delta^*$. Пусть $\omega \in \Delta^*$. Рассмотрим множество $W^\prime=\{(\omega_1,\omega_2)|\omega_1,\omega_2 \in \Delta^*, (a,b) \in P, \omega = \omega_1ab\omega_2\}$. Если $W^\prime$ пусто, то $\mu(\omega) = \omega$, иначе среди всех $\mu(\omega) = \mu(\omega_1\omega_2)$, где $|\omega_1|$ минимальна по всем парам $(\omega_1,\omega_2)\in W^\prime$. Назовём $\mu$ \textit{стирающим отображением}.
}

\mydefinition{
	\textit{L-графом} назовём восьмёрку\\ $G=(V,E,\Sigma,\Sigma_),\Sigma_(,P,S,F)$, где
	\begin{itemize}
		\item $V$ --- множество вершин, $S \in V$ - начальная вершина, $F \subseteq V$ --- множество заключительных вершин;

		\item $\Sigma$ - алфавит входных символов, $\Sigma_($ и $\Sigma_)$ - непересекающиеся алфавиты, $P$ --- \textit{D}-множество над $\Sigma_($ и $\Sigma_)$;

		\item множество $E$ описывает дуги и их символьные пометки: $E \subseteq \{(v_1,v_2,\alpha,\beta)|v_1,v_2\in V, \alpha \in \Sigma \cup \{\epsilon\},\beta \in \Sigma_( \cup \Sigma_)\}$.
	\end{itemize}
}
% section (end)
\section{Понятие ядра L-графа} % (fold)
%\label{sect2_2}
\mydefinition{
	Основные понятие, такие как \textit{Sentences}, \textit{sCore}, введем как у Касимовой.
}
% section (end)
\section{Понятие памяти L-графа} % (fold)

\mydefinition{
	Памятью \LGraphа $G(V,E,\Sigma,\Sigma_(,\Sigma_),P,S,F)$ будем называть пару $(v,\gamma)$, где
	\begin{itemize}
		\item $v \in V$ - вершина графа G,
		\item $\gamma \in \Sigma^*$ - цепочка из входных символов.
	\end{itemize}
}
\mydefinition{
	Будем говорить, что в маршруте $T$ \textit{фигурирует} память $m=(v,\gamma)$, если $T$ можно представить в виде $T = T_1T_2$, где $end(T_1) = v, \mu(T_1) = \gamma$. Обозначим это соотношение как $HasMemory(T,m)$. Если в этом определении $T_2 = \epsilon$, то будем говорить, что \textit{память маршрута $T$ равна $m$} ($Mem(T) = m$).
}

\mydefinition{
	\textit{Множеством (w,d)-остовных памятей} \LGraphа G будем называть множество $CoreMem(G,w,d) = \{(v,\gamma) | \exists T \in sCore(G,w,d): HasMemory(T,(v,\gamma))\}$.
}
%\label{sect2_3}

% section (end)
\clearpage