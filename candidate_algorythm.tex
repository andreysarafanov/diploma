\chapter{Построение ДКА-кандидата} \label{chapt3}

\section{Построение всех (w,d)-остовных памятей L-графа} 
\label{sect3_1}
\mydefinition{
	Будем говорить, что в маршруте $T$ \textit{фигурирует} память $m=(v,\gamma)$, если $T$ можно представить в виде $T = T_1T_2$, где $end(T_1) = v, \mu(T) = \gamma$. Обозначим это соотношение как $HasMemory(T,m)$. Если в этом определении $T_2 = \epsilon$, то будем говорить, что \textit{память $T$ равна $m$} ($Mem(T) = m$). \unsure{Или Mem два раза лучше не использовать?}
}
\\
\mydefinition{
	\textit{Множеством (w,d)-остовных памятей} \LGraphа G будем называть множество $Mem(G,w,d) = \{(v,\gamma) | \exists T \in sCore(G,w,d): HasMemory(T,(v,\gamma))\}$.
}
\\
\\

Тут должен быть алгоритм построения этого множества, но простой обход в глубину/ширину, естественно, не подходит. Наличие хотя бы одного цикла приведет к зацикливанию.

\section{Построение графа (w,d)-остовных памятей L-графа}
\label{sect3_2}
\mydefinition{
	\textit{Графом $(w,d)$-остовных памятей} \LGraphа $G(V,E^\prime,\Sigma,\Sigma_(,\Sigma_),P,S,F)$ назовём пятёрку $MemGraph(G,w,d) = (M,F,S,E,\Sigma)$, где
	\begin{itemize}
		\item $M = Mem(G,w,d)$ --- множество вершин графа, $S \in M$ - начальная вершина, $F \subseteq V$ --- множество заключительных вершин;

		\item $\Sigma$ - алфавит входных символов;

		\item множество $E$ описывает дуги и их символьные пометки: $E \subseteq \{(m_1,m_2,\alpha)|m_1,m_2\in M, \alpha \in \Sigma \cup \{\epsilon\}\}$.
	\end{itemize}
}
В роли вершин графа выступает множество $(w,d)$-остовных памятей \LGraphа $G$. Опишем алгоритм \unsure{создания} множества $E$.
Для каждой вершины $m_1=(v_1,\gamma_1) \in M$ рассмотрим поочередно все дуги из $E^\prime$ вида $(v_1, v_2, \alpha, \beta)$. Для каждой вершины $m_2=(v_2, \gamma_2) \in M$, такой что $\mu(\gamma_1\beta) = \gamma_2$ добавим в $E$ дугу $(m_1,m_2,\alpha)$. \unsure{Получилось криво, к тому же не ясно ещё, так ли стоит этот граф определять. Альтернативный вариант - добавлять дугу $(m_1,m_2,\alpha)$, если $\exists T \in sCore(G,w,d), T = T_1T_2T_3, mem(T_1) = m_1, mem(T_1T_2) = m_2, \omega(T_1T_2) = \omega(T_1)\beta$}

\section{Построение графа-кандидата}
На основе $M_1 = MemGraph(G,1,1)$ и $M_2 = MemGraph(G,2,2)$ построим граф недетерминированного конечного автомата $Cand(G)$.
\clearpage