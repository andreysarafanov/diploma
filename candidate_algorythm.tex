\chapter{Построение ДКА-кандидата} \label{chapt3}

\section{Построение всех (w,d)-остовных памятей L-графа} 
\label{sect3_1}
Тут должен быть алгоритм построения этого множества, но простой обход в глубину/ширину, естественно, не подходит. Наличие хотя бы одного цикла приведет к зацикливанию.

\section{Построение графа (w,d)-остовных памятей L-графа}
\label{sect3_2}

\mydefinition{
	\textit{Графом $(w,d)$-остовных памятей} \LGraphа $G(V,E,\Sigma,\Sigma_(,\Sigma_),P,S,F)$ назовём пятёрку $M(V_M,F_M,S_M,E_M,\Sigma_M)$, где
	\begin{itemize}
		\item $V_M = Mem(G,w,d)$ --- множество вершин графа,
		\item $S_M = (S,\epsilon) \in V_M$ - начальная вершина,
		\item $F = \{(v,m) \in V_M|v \in F\}$ --- множество заключительных вершин,
		\item $\Sigma$ - алфавит входных символов,
		\item множество $E_M$ описывает дуги и их символьные пометки: $E_M = \{((v_1,\gamma_1),(v_2, \gamma_2),\alpha)|(v_1,\gamma_1) \in V_M,(v_2, \gamma_2)\in V_M, \alpha \in \Sigma \cup \{\epsilon\}, \exists \beta \in \Sigma_( \cup \Sigma_): \mu(\gamma_1\beta) = \gamma_2, (v_1,v_2,\alpha,\beta) \in E\}$.
	\end{itemize}
	Будем обозначать граф $(w,d)$-остовных памятей графа $G$ как $MemGraph(G,w,d)$.
}

\section{Построение графа-кандидата}
Опишем построение на основе детерминированного \LGraphа $G(V,E,\Sigma,\Sigma_(,\Sigma_),P,S,F)$ графа недетерминированного конечного автомата $Cand(G)$.

Пусть $M_1(V_{M_1},F_{M_1},S_M,E_{M_1},\Sigma) = MemGraph(G,1,1)$ и $M_2(V_{M_2},F_{M_2},S_M,E_{M_2},\Sigma) = MemGraph(G,2,2)$. 
Введём отображение $PrevMemory: V_{M_2} \rightarrow V_{M_1} \cup V_{M_2} $
Обозначим $ToRemove(G) = \{m|m \in V_{M_2}, m \not\in V_{M_1}\}$. 

\clearpage