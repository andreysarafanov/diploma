\chapter{Введение}
\label{chapt_intro}
Формальным языком называется множество цепочек конечной длины, состоящих из элементов некоторого непустого множества $\Sigma$. В этом случае говорят, что язык построен над алфавитом $\Sigma$. Цепочку нулевой длины обозначают символом $\epsilon$. В теории формальных языков классической считается классификация формальных языков, основанная на т.н. иерархии Хомского, предложенной американским лингвистом Ноамом Хомским. Согласно ей, формальные языки делятся на 4 класса:
\begin{enumerate}[start=0]
	\item Рекурсивно перечислимые (неограниченные)
	\item Контекстно-зависимые
	\item Контекстно-свободные (бесконтекстные)
	\item Регулярные
\end{enumerate}
Стоит также отдельно отметить выделяемые в классе 2 подклассы детерминированых и недетерминированных контекстно-свободных языков.\\
Существует множество способов описания формальных грамматик, из них наиболее распространенными являются формальные грамматики и абстрактные вычислительные устройства, такие как машина Тьюринга, магазинные автоматы, конечные автоматы \unsure{(скопировал у Касимовой)}.
\\
Формальные грамматики применяются для описания формальных языков всех классов и представляют из себя четверки вида $(N,\Sigma,P,S)$, где
\begin{itemize}
	\item $N$ - конечное множество нетерминальных символов,
	\item $\Sigma$ - конечное множество терминальных символов,
	\item $P$ - конечное множество правил вывода вида $Left \rightarrow Right, Left \in (N \cup \Sigma)^+, Right \in (N \cup \Sigma)^*$,
	\item $S \in N$ - начальный символ.
\end{itemize}  
\clearpage