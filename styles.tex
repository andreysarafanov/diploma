%%% Макет страницы %%%
\geometry{a4paper,top=2cm,bottom=2cm,left=3cm,right=1cm}

%%% Кодировки и шрифты %%%
\renewcommand{\rmdefault}{ftm} % Включаем Times New Roman

%%% Выравнивание и переносы %%%
\sloppy					% Избавляемся от переполнений
\clubpenalty=10000		% Запрещаем разрыв страницы после первой строки абзаца
\widowpenalty=10000		% Запрещаем разрыв страницы после последней строки абзаца

%%% Библиография %%%
\makeatletter
\bibliographystyle{utf8gost705u}	% Оформляем библиографию в соответствии с ГОСТ 7.0.5
\renewcommand{\@biblabel}[1]{#1.}	% Заменяем библиографию с квадратных скобок на точку:
\makeatother

%%% Изображения %%%
\graphicspath{{images/}} % Пути к изображениям

%%% Цвета гиперссылок %%%
\definecolor{linkcolor}{rgb}{0.9,0,0}
\definecolor{citecolor}{rgb}{0,0.6,0}
\definecolor{urlcolor}{rgb}{0,0,1}
\hypersetup{
    colorlinks, linkcolor={linkcolor},
    citecolor={citecolor}, urlcolor={urlcolor}
}
%%% Из дефолтного определения функции убрали добавление строки ГЛАВА ТАКАЯ-ТО %%%
\makeatletter
\renewcommand\@makechapterhead[1]{%
    \vspace*{50\p@}%
    {
        \parindent \z@ \raggedright \normalfont
        \interlinepenalty\@M
        \Huge \bfseries #1\par\nobreak
       \vskip 40\p@
    }
}
\makeatother

%%% Оглавление %%%
\renewcommand{\cftchapdotsep}{\cftdotsep}

%%% Нумерация определений %%%
\newcounter{DefCounter}
\setcounter{DefCounter}{0}

\newcommand{\mydefinition}[1]{
	\stepcounter{DefCounter}
	\noindent
	\textbf{Определение\,\theDefCounter.} #1
}

\newcommand{\unsure}[1]{
	\textcolor{red}{#1}
}